% vim: set nospell fdm=marker foldmarker=%[,%] filetype=tex :
\documentclass[%[1
	%12pt,
	%oneside,
	%margepetite,
	%margegrande
]{EcoleLecture}
\titre[TitreCourt]{TitreTrèsLong}
\sujet[SujetCourt]{VersionLonguesDuSujet}
\createur[CréateurCourt]{CréateurQuiMetLeDocumentEnPage}
\auteur[AuteurCourt]{AuteurOriginelDuTexteProduit}
\mdate{Date}
\producteur{Producteur}
\motscles{MotsClés}
\begin{document}%[1
\maketitle

Texte qui ne veut rien dire mais qui est bien utile pour aider à la mise en page du document. C'est ce que l'on appelle couramment du \emph{gris typographique}.

\begin{colonnes}
	Du texte en deux colonnes dont les lignes sont numérotées.
\end{colonnes}


\end{document}
